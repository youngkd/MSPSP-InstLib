% Data61 Summer Internship
% DataZinc Format Description
% Kenneth Young

\documentclass[12pt]{article}

\usepackage[english]{babel}
\usepackage{latexsym,amssymb,amsmath,epsfig,amsfonts,graphicx,url,verbatim}
\usepackage{float}
%\usepackage{xcolor}
\usepackage[pdftex,usenames,dvipsnames]{color}
\usepackage{multicol,multirow}
\usepackage{authblk}
\usepackage{chngpage}
\usepackage{mathrsfs}
\usepackage{listings}
\usepackage{booktabs}
\usepackage{subfig}
\usepackage{xspace}

\definecolor{ForestGreen}{RGB}{34,106,46}

\setlength{\topmargin}{-15pt}
\setlength{\headsep}{20pt}
\setlength{\headheight}{14.5pt}
\setlength{\textheight}{600pt}
\setlength{\oddsidemargin}{0pt}
\setlength{\evensidemargin}{0pt}
\setlength{\textwidth}{460pt}
\setlength{\parskip}{.30cm}
\parskip=10pt

\interdisplaylinepenalty=1000


%----------------------------------------------%
% Syntax highlighting for MiniZinc in listings %
%----------------------------------------------%

\definecolor{lightgray}{rgb}{0.97, 0.97, 0.97}

\lstdefinelanguage{minizinc}{
    morekeywords={
        %% MiniZinc keywords
        %%
        ann, annotation, any, array, assert,
        bool,
        constraint,
        else, elseif, endif, enum, exists,
        float, forall, function,
        if, in, include, int,
        list,
        minimize, maximize,
        of, op, output,
        par, predicate,
        record,
        set, solve, string,
        test, then, tuple, type,
        var,
        where,
        %% MiniZinc functions
        %%
        abort, abs, acosh, array_intersect, array_union,
        array1d, array2d, array3d, array4d, array5d, array6d, asin, assert, atan,
        bool2int,
        card, ceil, combinator, concat, cos, cosh,
        dom, dom_array, dom_size, dominance,
        exp,
        fix, floor,
        index_set, index_set_1of2, index_set_2of2, index_set_1of3, index_set_2of3, index_set_3of3,
        int2float, is_fixed,
        join,
        lb, lb_array, length, let, ln, log, log2, log10,
        min, max,
        pow, product,
        round,
        set2array, show, show_int, show_float, sin, sinh, sqrt, sum,
        tan, tanh, trace,
        ub, and ub_array,
        %% Search keywords
        %%
        bool_search, int_search, seq_search, priority_search,
        %% MiniSearch keywords
        %%
        minisearch, search, while, repeat, next, commit, print, post, sol, scope, time_limit, break, fail
    },
    sensitive=true, % are the keywords case sensitive
    morecomment=[l][\em\color{ForestGreen}]{\%},
    %morecomment=[s]{/*}{*/},
    morestring=[b]",
}

%% Settings for listings
%%
\lstset{ %
    backgroundcolor=\color{lightgray},  % choose the background color; you must add
                                        % \usepackage{color} or \usepackage{xcolor}
    basicstyle=\scriptsize\ttfamily,    % the size of the fonts that are used for the code
    belowskip=-2em,
    breakatwhitespace=false,            % sets if automatic breaks should only happen at whitespace
    breaklines=true,                    % sets automatic line breaking
    captionpos=b,                       % sets the caption-position to bottom
    commentstyle=\color{ForestGreen},   % comment style
    %deletekeywords={...},              % if you want to delete keywords from the given language
    escapeinside={\%*}{*)},             % if you want to add LaTeX within your code
    extendedchars=true,                 % lets you use non-ASCII characters; for 8-bits
                                        % encodings only, does not work with UTF-8
    frame=single,                       % adds a frame around the code
    keepspaces=true,                    % keeps spaces in text, useful for keeping indentation
                                        % of code (possibly needs columns=flexible)
    keywordstyle=\bfseries\color{blue}, % keyword style
    language=minizinc,                  % the language of the code
    %morekeywords={*,...},              % if you want to add more keywords to the set
    numbers=none,                       % where to put the line-numbers; possible values are (none, left, right)
    %numbersep=5pt,                     % how far the line-numbers are from the code
    %numberstyle=\tiny\color{Gray},     % the style that is used for the line-numbers
    rulecolor=\color{black},            % if not set, the frame-color may be changed
                                        % on line-breaks within not-black text (e.g. comments (green here))
    showspaces=false,                   % show spaces everywhere adding particular
                                        % underscores; it overrides 'showstringspaces'
    showstringspaces=false,             % underline spaces within strings only
    showtabs=false,                     % show tabs within strings adding particular underscores
    %stepnumber=1,                      % the step between two line-numbers. If it's 1, each line will be numbered
    stringstyle=\color{Red},            % string literal style
    tabsize=2,                          % sets default tabsize to 2 spaces
    title=\lstname                      % show the filename of files included with \lstinputlisting;
                                        % also try caption instead of title
}

%\def\mzninline{\lstinline[basicstyle=\ttfamily,annotationstyle=\normalfont]}
\def\mzninline{\verb}


\begin{document}

\section{DataZinc Format Description}

This document describes the formatting of one of the DataZinc (dzn) files
that store all the data required to fully specify an instance of
the MSPSP.

Each dzn file contains the following:
\begin{itemize}
    \item {\tt seed}: Seed used for the randomisation. Integer between 1,000,000 and 9,999,999.
    \item {\tt mint}: Lower bound on the total project duration. Na\"{i}vely calculated using 
        the length of the critical path.
    \item {\tt maxt}: Upper bound on the total project duration. Na\"{i}vely calculated using
        as the sum of all the processing times.
    \item {\tt nActs}: Total number of activites. Includes the dummy start and end activities.
    \item {\tt dur}: Duration/processing time of each activity. Stored as an $1\times n$ array.
    \item {\tt nSkills}: Total number of skills.
    \item {\tt sreq}: Skill requirement for each activity. Stored as an $n\times l$ array. If entry
    $(i,j)=0$ then activity $i$ does not require any resources mastering skill $j$.
    \item {\tt nResources}: Total number of resources.
    \item {\tt mastery}: Skills each resource has mastered. Stored as an $m\times l$ array containing
    either {\tt true} or {\tt false}.
    \item {\tt nPrecs}: Total number of precedence relations. Define this as $(|E|)$.
    \item {\tt pred}: A $1\times |E|$ array storing the predecessor for each precedence relation.
    \item {\tt succ}: A $1\times |E|$ array storing the successor for each precedence relation.
\end{itemize}

The following items are pieces of data that can be derived from the basic 
data specified above.
We decided to compute these in a preprocessing step and store it in the data file to 
improve the solve times.
\begin{itemize}
    \item {\tt nUnrels}: Total number of unrelated activity pairs. Define this as $(|U|)$.
    \item {\tt unpred}: A $1\times |U|$ array storing the first activity for each unrelated pair.
    \item {\tt unsucc}: A $1\times |U|$ array storing the second activity for each unrelated pair.
    \item {\tt USEFUL\_RES}: A $n\times 1$ array storing the set of useful resources for each activity.
        I.e. the set of all resources mastering a skill this activities requires.
    \item {\tt POTENTIAL\_ACT}: A $m\times 1$ array storing the set of potential activities for each resource.
        I.e. the set of all activities that each resource could feasibly be assigned.
\end{itemize}

Lastly, the instances of set 2 and 3 also store one extra piece of derived data:
\begin{itemize}
    \item {\tt SumOfsreq}: Sum of the skill requirements of all activities. This can
        be useful for doing analysis on the complexity of an instance.
\end{itemize}

We now give a simple example of a DataZinc file with 8 activities, 2 skills and 7 resources:\\[3pt]
File path = {\tt /instances/set-3/set-3a/}\\[3pt]
File name = {\tt inst\_set3a\_sf0\_nc1.8\_n6\_l2\_m7\_00.dzn}\\[3pt]
Contents:

\begin{lstlisting}[language=minizinc]
% seed = 0

mint = 14;
% maxt = 24;

nActs = 8;
dur = [0,2,4,2,8,4,4,0];

nSkills = 2;
sreq = [| 0,0,
    | 0,3,
    | 1,2,
    | 2,0,
    | 1,0,
    | 3,0,
    | 1,2,
    | 0,0, |];

% SumOfsreq = 15;

nResources = 7;
mastery = [| true,true,
    | false,true,
    | false,true,
    | true,false,
    | true,false,
    | true,true,
    | false,true, |];

nPrecs = 11;
pred = [1,2,2,2,3,3,4,4,5,6,7];
succ = [2,3,5,6,4,6,6,7,7,8,8];

nUnrels = 4;
unpred = [3,4,5,6];
unsucc = [5,5,6,7];

USEFUL_RES = [{},
    {1,2,3,6,7},
    {1,2,3,4,5,6,7},
    {1,4,5,6},
    {1,4,5,6},
    {1,4,5,6},
    {1,2,3,4,5,6,7},
    {}];

POTENTIAL_ACT = [{2,3,4,5,6,7},
    {2,3,7},
    {2,3,7},
    {3,4,5,6,7},
    {3,4,5,6,7},
    {2,3,4,5,6,7},
    {2,3,7}];
\end{lstlisting}

\end{document}
